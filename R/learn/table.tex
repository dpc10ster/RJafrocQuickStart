% Options for packages loaded elsewhere
\PassOptionsToPackage{unicode}{hyperref}
\PassOptionsToPackage{hyphens}{url}
%
\documentclass[
]{article}
\usepackage{lmodern}
\usepackage{amssymb,amsmath}
\usepackage{ifxetex,ifluatex}
\ifnum 0\ifxetex 1\fi\ifluatex 1\fi=0 % if pdftex
  \usepackage[T1]{fontenc}
  \usepackage[utf8]{inputenc}
  \usepackage{textcomp} % provide euro and other symbols
\else % if luatex or xetex
  \usepackage{unicode-math}
  \defaultfontfeatures{Scale=MatchLowercase}
  \defaultfontfeatures[\rmfamily]{Ligatures=TeX,Scale=1}
\fi
% Use upquote if available, for straight quotes in verbatim environments
\IfFileExists{upquote.sty}{\usepackage{upquote}}{}
\IfFileExists{microtype.sty}{% use microtype if available
  \usepackage[]{microtype}
  \UseMicrotypeSet[protrusion]{basicmath} % disable protrusion for tt fonts
}{}
\makeatletter
\@ifundefined{KOMAClassName}{% if non-KOMA class
  \IfFileExists{parskip.sty}{%
    \usepackage{parskip}
  }{% else
    \setlength{\parindent}{0pt}
    \setlength{\parskip}{6pt plus 2pt minus 1pt}}
}{% if KOMA class
  \KOMAoptions{parskip=half}}
\makeatother
\usepackage{xcolor}
\IfFileExists{xurl.sty}{\usepackage{xurl}}{} % add URL line breaks if available
\IfFileExists{bookmark.sty}{\usepackage{bookmark}}{\usepackage{hyperref}}
\hypersetup{
  pdftitle={Untitled},
  pdfauthor={DPC},
  hidelinks,
  pdfcreator={LaTeX via pandoc}}
\urlstyle{same} % disable monospaced font for URLs
\usepackage[margin=1in]{geometry}
\usepackage{color}
\usepackage{fancyvrb}
\newcommand{\VerbBar}{|}
\newcommand{\VERB}{\Verb[commandchars=\\\{\}]}
\DefineVerbatimEnvironment{Highlighting}{Verbatim}{commandchars=\\\{\}}
% Add ',fontsize=\small' for more characters per line
\usepackage{framed}
\definecolor{shadecolor}{RGB}{248,248,248}
\newenvironment{Shaded}{\begin{snugshade}}{\end{snugshade}}
\newcommand{\AlertTok}[1]{\textcolor[rgb]{0.94,0.16,0.16}{#1}}
\newcommand{\AnnotationTok}[1]{\textcolor[rgb]{0.56,0.35,0.01}{\textbf{\textit{#1}}}}
\newcommand{\AttributeTok}[1]{\textcolor[rgb]{0.77,0.63,0.00}{#1}}
\newcommand{\BaseNTok}[1]{\textcolor[rgb]{0.00,0.00,0.81}{#1}}
\newcommand{\BuiltInTok}[1]{#1}
\newcommand{\CharTok}[1]{\textcolor[rgb]{0.31,0.60,0.02}{#1}}
\newcommand{\CommentTok}[1]{\textcolor[rgb]{0.56,0.35,0.01}{\textit{#1}}}
\newcommand{\CommentVarTok}[1]{\textcolor[rgb]{0.56,0.35,0.01}{\textbf{\textit{#1}}}}
\newcommand{\ConstantTok}[1]{\textcolor[rgb]{0.00,0.00,0.00}{#1}}
\newcommand{\ControlFlowTok}[1]{\textcolor[rgb]{0.13,0.29,0.53}{\textbf{#1}}}
\newcommand{\DataTypeTok}[1]{\textcolor[rgb]{0.13,0.29,0.53}{#1}}
\newcommand{\DecValTok}[1]{\textcolor[rgb]{0.00,0.00,0.81}{#1}}
\newcommand{\DocumentationTok}[1]{\textcolor[rgb]{0.56,0.35,0.01}{\textbf{\textit{#1}}}}
\newcommand{\ErrorTok}[1]{\textcolor[rgb]{0.64,0.00,0.00}{\textbf{#1}}}
\newcommand{\ExtensionTok}[1]{#1}
\newcommand{\FloatTok}[1]{\textcolor[rgb]{0.00,0.00,0.81}{#1}}
\newcommand{\FunctionTok}[1]{\textcolor[rgb]{0.00,0.00,0.00}{#1}}
\newcommand{\ImportTok}[1]{#1}
\newcommand{\InformationTok}[1]{\textcolor[rgb]{0.56,0.35,0.01}{\textbf{\textit{#1}}}}
\newcommand{\KeywordTok}[1]{\textcolor[rgb]{0.13,0.29,0.53}{\textbf{#1}}}
\newcommand{\NormalTok}[1]{#1}
\newcommand{\OperatorTok}[1]{\textcolor[rgb]{0.81,0.36,0.00}{\textbf{#1}}}
\newcommand{\OtherTok}[1]{\textcolor[rgb]{0.56,0.35,0.01}{#1}}
\newcommand{\PreprocessorTok}[1]{\textcolor[rgb]{0.56,0.35,0.01}{\textit{#1}}}
\newcommand{\RegionMarkerTok}[1]{#1}
\newcommand{\SpecialCharTok}[1]{\textcolor[rgb]{0.00,0.00,0.00}{#1}}
\newcommand{\SpecialStringTok}[1]{\textcolor[rgb]{0.31,0.60,0.02}{#1}}
\newcommand{\StringTok}[1]{\textcolor[rgb]{0.31,0.60,0.02}{#1}}
\newcommand{\VariableTok}[1]{\textcolor[rgb]{0.00,0.00,0.00}{#1}}
\newcommand{\VerbatimStringTok}[1]{\textcolor[rgb]{0.31,0.60,0.02}{#1}}
\newcommand{\WarningTok}[1]{\textcolor[rgb]{0.56,0.35,0.01}{\textbf{\textit{#1}}}}
\usepackage{longtable,booktabs}
% Correct order of tables after \paragraph or \subparagraph
\usepackage{etoolbox}
\makeatletter
\patchcmd\longtable{\par}{\if@noskipsec\mbox{}\fi\par}{}{}
\makeatother
% Allow footnotes in longtable head/foot
\IfFileExists{footnotehyper.sty}{\usepackage{footnotehyper}}{\usepackage{footnote}}
\makesavenoteenv{longtable}
\usepackage{graphicx}
\makeatletter
\def\maxwidth{\ifdim\Gin@nat@width>\linewidth\linewidth\else\Gin@nat@width\fi}
\def\maxheight{\ifdim\Gin@nat@height>\textheight\textheight\else\Gin@nat@height\fi}
\makeatother
% Scale images if necessary, so that they will not overflow the page
% margins by default, and it is still possible to overwrite the defaults
% using explicit options in \includegraphics[width, height, ...]{}
\setkeys{Gin}{width=\maxwidth,height=\maxheight,keepaspectratio}
% Set default figure placement to htbp
\makeatletter
\def\fps@figure{htbp}
\makeatother
\setlength{\emergencystretch}{3em} % prevent overfull lines
\providecommand{\tightlist}{%
  \setlength{\itemsep}{0pt}\setlength{\parskip}{0pt}}
\setcounter{secnumdepth}{5}

\title{Untitled}
\author{DPC}
\date{11/25/2020}

\begin{document}
\maketitle

{
\setcounter{tocdepth}{2}
\tableofcontents
}
\begin{Shaded}
\begin{Highlighting}[]
\KeywordTok{set.seed}\NormalTok{(}\DecValTok{123}\NormalTok{)}
\NormalTok{testtable =}\StringTok{ }\KeywordTok{data.frame}\NormalTok{(}\KeywordTok{matrix}\NormalTok{(}\KeywordTok{paste0}\NormalTok{(}\KeywordTok{sample}\NormalTok{(}\DecValTok{1}\OperatorTok{:}\DecValTok{100}\NormalTok{,}\DecValTok{16}\NormalTok{, }\DataTypeTok{replace =}\NormalTok{ T), }\StringTok{"\%"}\NormalTok{), }\DataTypeTok{nrow =} \DecValTok{4}\NormalTok{, }\DataTypeTok{byrow =}\NormalTok{ T), }\DataTypeTok{stringsAsFactors =}\NormalTok{ F)}
\NormalTok{testtable}\OperatorTok{$}\NormalTok{group =}\StringTok{ }\KeywordTok{c}\NormalTok{(}\StringTok{"Var1"}\NormalTok{, }\StringTok{"Var2"}\NormalTok{, }\StringTok{"Var3"}\NormalTok{, }\StringTok{"Var4"}\NormalTok{)}
\NormalTok{testtable =}\StringTok{ }\NormalTok{testtable[}\KeywordTok{c}\NormalTok{(}\DecValTok{5}\NormalTok{,}\DecValTok{1}\OperatorTok{:}\DecValTok{4}\NormalTok{)]}
\KeywordTok{colnames}\NormalTok{(testtable) =}\StringTok{ }\KeywordTok{c}\NormalTok{(}\StringTok{"Variable"}\NormalTok{,}\StringTok{"All}\CharTok{\textbackslash{}n}\StringTok{(n = 300)"}\NormalTok{, }\StringTok{"Group1}\CharTok{\textbackslash{}n}\StringTok{(n = 120)"}\NormalTok{, }\StringTok{"Group2}\CharTok{\textbackslash{}n}\StringTok{(n = 100)"}\NormalTok{, }\StringTok{"Group3}\CharTok{\textbackslash{}n}\StringTok{(n = 80)"}\NormalTok{)}
\end{Highlighting}
\end{Shaded}

\begin{Shaded}
\begin{Highlighting}[]
\NormalTok{    testtable }\OperatorTok{\%\textgreater{}\%}\StringTok{ }\KeywordTok{mutate\_all}\NormalTok{(linebreak) }\OperatorTok{\%\textgreater{}\%}\StringTok{ }
\StringTok{    }\KeywordTok{kable}\NormalTok{(}\DataTypeTok{format =} \StringTok{"latex"}\NormalTok{, }\DataTypeTok{align =} \KeywordTok{c}\NormalTok{(}\StringTok{"l"}\NormalTok{, }\StringTok{"c"}\NormalTok{, }\StringTok{"c"}\NormalTok{, }\StringTok{"c"}\NormalTok{, }\StringTok{"c"}\NormalTok{), }\DataTypeTok{caption =} \StringTok{"Caption"}\NormalTok{, }\DataTypeTok{booktabs =}\NormalTok{ T, }\DataTypeTok{col.names =} \KeywordTok{linebreak}\NormalTok{(}\KeywordTok{names}\NormalTok{(testtable), }\DataTypeTok{align =} \StringTok{"c"}\NormalTok{)) }\OperatorTok{\%\textgreater{}\%}
\StringTok{      }\KeywordTok{kable\_styling}\NormalTok{(}\DataTypeTok{position =} \StringTok{"center"}\NormalTok{, }\DataTypeTok{latex\_options =} \KeywordTok{c}\NormalTok{(}\StringTok{"hold\_position"}\NormalTok{)) }\OperatorTok{\%\textgreater{}\%}
\StringTok{      }\KeywordTok{footnote}\NormalTok{(}\DataTypeTok{general =} \StringTok{"* This is a note to show what * shows in this table plus some addidtional words to make this string a bit longer. Still a bit more"}\NormalTok{, }\DataTypeTok{threeparttable =}\NormalTok{ T, }\DataTypeTok{general\_title =} \StringTok{"Anmerkung:"}\NormalTok{, }\DataTypeTok{title\_format =} \StringTok{"italic"}\NormalTok{)}
\end{Highlighting}
\end{Shaded}

\begin{table}[!h]

\caption{\label{tab:unnamed-chunk-3}Caption}
\centering
\begin{threeparttable}
\begin{tabular}[t]{lcccc}
\toprule
Variable & \textbackslash{}makecell[c]\{All\textbackslash{}\textbackslash{}(n = 300)\} & \textbackslash{}makecell[c]\{Group1\textbackslash{}\textbackslash{}(n = 120)\} & \textbackslash{}makecell[c]\{Group2\textbackslash{}\textbackslash{}(n = 100)\} & \textbackslash{}makecell[c]\{Group3\textbackslash{}\textbackslash{}(n = 80)\}\\
\midrule
Var1 & 31\% & 79\% & 51\% & 14\%\\
Var2 & 67\% & 42\% & 50\% & 43\%\\
Var3 & 14\% & 25\% & 90\% & 91\%\\
Var4 & 69\% & 91\% & 57\% & 92\%\\
\bottomrule
\end{tabular}
\begin{tablenotes}
\item \textit{Anmerkung:} 
\item * This is a note to show what * shows in this table plus some addidtional words to make this string a bit longer. Still a bit more
\end{tablenotes}
\end{threeparttable}
\end{table}

\end{document}
